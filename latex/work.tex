
\documentclass{article}
\usepackage{glossaries}
\usepackage{cite}
\usepackage{color}
\usepackage[english]{babel}
\usepackage{filecontents}
\usepackage[dvipsnames]{xcolor}
\usepackage[numbers,sort&compress]{natbib}
\usepackage{ifthen}
\usepackage{pdfpages}
\usepackage{listings}

\makeglossaries

\newglossaryentry{migration}{
    name={migration},
    description={Migration is the ability to switch to a new version, platform, or physical location \cite{what-is-migration}. This term come from database management, but is also used in other computer science fields.
    }
}

\newglossaryentry{dns}{
    name={DNS},
    description={Domain Name System, used to resolve internet addresses into human readable domain names}}

\newglossaryentry{partition}{
    name={partition},
    description={Bill Calkins defines partition as "Disks are divided into regions called “disk slices” or “disk partitions.” A slice is composed of a single range of contiguous blocks. It is a physical subset of the disk (except for slice 2, which represents the entire disk)."\cite{what_is_partition} }
}

\newglossaryentry{virtual_machine}{
    name={virtual machine},
    description={Oracle Corporation defines it "as a virtualized operating system with its associated software and applications"\cite{what_is_virtual_machine}
    }
}

\newglossaryentry{IT}{
    name={I.T.},
    description={or Internet Technology is defined by Techopedia as use of computers, storage networking and other devices and a variety of other hardware with the purpose of exchanging digital information \cite{techopedia-it-definition}
    }
}

\newglossaryentry{standup}
{
    name=stand-up,
    description={A daily team meeting where every team member gives an overview of done tasks, current tasks and any future work that needs to be done by that member. }
}

\newglossaryentry{bash}
{
    name=bash,
    description={Bourne Again SHell, a UNIX command line virtual shell language}
}

\newglossaryentry{open-source}
{
    name={open-source},
    description={Software that makes its code openly available, allows derived work without restrictions and its distribution is done freely}
}

\newglossaryentry{operating-system}
{
    name={operating system},
    description={The platform on which all user and system software is running. Examples are Windows, Linux, OSX, Android, iOSX}
}

\newglossaryentry{ssh}{
    name={ssh},
    description={
        SSH: secure shell, a network protocol that allows a remote connection to a terminal
    }
}

\newglossaryentry{natnetwork}{
    name={NAT network},
    description={
        A technique where one IP address is mapped to more internal ip addresses, used for security and reduced number of ips. \cite{whatisnat}
    }
}

\newglossaryentry{ipaddress}{
    name={IP address},
    description={
        An identifier that helps locating a resource on the network, usually represents a computer device connected to a network
    }
}

\newglossaryentry{api}{
    name={API},
    description = {
        Application Programmable interface, usually a way for exposing features of a certain program to the developer. Often used to automate a task.
    }
}

\newglossaryentry{vagrant}{
    name={Vagrant},
    description = {
        Vagrant provides easy to configure, reproducible, and portable work environments built on top of industry-standard technology and controlled by a single consistent workflow to help maximize the productivity and flexibility of you and your team. \cite{vagrant-definition}
    }
}

\newglossaryentry{puppet}{
    name={puppet},
    description={
        Open source Puppet helps you describe machine configurations in a declarative language, bring machines to a desired state, and keep them there through automation.\cite{puppet-definition}
    }
}

\newglossaryentry{ruby-on-rails}{
    name={Ruby On Rails},
    description={
        Rails is a web application development framework written in the Ruby language. It is designed to make programming web applications easier by making assumptions about what every developer needs to get started. It allows you to write less code while accomplishing more than many other languages and frameworks.\cite{ruby-on-rails-definition}
    }
}

\newglossaryentry{ruby}{
    name={ruby},
    description={
        A dynamic, open source programming language with a focus on simplicity and productivity. It has an elegant syntax that is natural to read and easy to write.\cite{ruby-definition}
    }
}

\newglossaryentry{html}{
    name={html},
    description={
        Hypertext Markup Language, used for displaying web page
    }
}

\newglossaryentry{ram}{
    name={RAM},
    description={Random Access Memory, also main memory, is a computer hardware component that stores information for fast retrieval.}
}

\newglossaryentry{cpu}{
    name={CPU},
    description={Central Processing Unit, the hardware component that is responsible for executing instructions, evaluate expressions and perform algebraic operations.}
}

\newglossaryentry{gpu}{
    name={GPU},
    description={Graphical Processing Unit, a computer hardware component that specialises in processing highly parallel instructions.}
}

\newglossaryentry{kernel}{
    name={kernel},
    description={Is the core component of an operating system that provides all the services necessary for every component of the system.}
}

\newglossaryentry{vsphere}{
    name={vSphere},
    description={ Is a technology used to manage large collections of infrastructure (such as CPUs, storage, and networking) as a seamless and dynamic operating environment, and also manages the complexity of a datacenter\cite{vsphere-definition}
    }
}

\newglossaryentry{xeon}{
    name={Xeon},
    description={ The Xeon processor family is Intel's server grade processors
    }
}

\newglossaryentry{vmware}{
    name={VmWare},
    description={VMware is the leader in so-called virtual machine software, it is also the name of the product that the company produces \cite{vmware-definition}
    }
}

\newglossaryentry{virtualbox}
{name={virtualbox},
    description={Open-source virtualisation technology created by Oracle}
}

\newglossaryentry{amazon-web-services}{
    name={Amazon Web Services},
    description={Amazon's cloud infrastructure as a service
    }
}

\newglossaryentry{azure}{
    name={Azure},
    description={Microsoft's cloud infrastructure as a service
    }
}








\newglossaryentry{gem}{
    name={gem},
    description={Ruby programming language term for module or package
    }
}

\newglossaryentry{distribution}{name={distribution},
    description={Is a term that describes a Linux operating system with associated packages and pre-installed software
    }
}

\newglossaryentry{netmask}{name={network mask},
    description={A netmask is a 32 bits field whose p high order bits are set to 1 and the low order bits are set to 0. The number of high order bits set 1 indicates the length of the subnet identifier. Netmasks are usually represented in the same dotted decimal format as IPv4 addresses. \cite[p.143]{netmask-definition}
    }
}


\let\oldcite=\cite
\renewcommand\cite[1]{\ifthenelse{\equal{#1}{_NEEDED_}}{[citation~needed]}{\oldcite{#1}}}
\renewcommand*{\glstextformat}[1]{\textcolor{blue}{#1}}
% ENDCUSTOMMACROS

\definecolor{Mycolor2}{HTML}{4a4a4a}

\begin{document}
\section{Work}
The first component of the work is a ruby module designed specially to act as a back-end for the entire solution. A module is a collection of programming functions and libraries that are bundled together to achieve a particular task. In \gls{ruby}, modules are called gems. For writing my first \gls{gem}, the official ruby guide was used as guide \cite{ruby-official-guide}. 

The file \texttt{lib/deeploy.rb} is what sets up the database connection and bootstraps the whole module by auto-loading every component. To ease integration, the file loads configuration properties from the file \texttt{lib/config/application.yml}. \texttt{Application.yml} contains configuration for production, development and testing, this is to allow separation and configuration of different parameters. The file specifies location of the database file when in development, or it has the database drivers in production with information about connection parameters. The file also describes the network interfaces to be used in each environment.

\texttt{lib/deeploy.rb} has a helper method that returns the current network interface and \gls{netmask}, this function is responsible for generating \glspl{ipaddress} and also for verifying that the virtual network is up and running. It does so by issuing commands for bringing up the virtual networks \texttt{vboxnet} up, then using network sockets to extract relevant information. This function is the programatic equivalent to the command \texttt{ifconfig or ipconfig}.

The file also contains a static function that returns a map of all supported distributions and the associated \gls{vagrant} machine. The function is helpful when validating creation, and will error out if the \gls{distribution} name is not in the list. When extending the supported distribution, this function needs updating. 

Similarly to the supported \glspl{distribution} function, there is a static function to return the supported packages list, used for displaying the options and validating unsupported packages.

The module provides a function slugify, its purpose is to convert string with special characters and spaces to a set of words and dashes, non-ASCII characters are stripped.

\gls{ipaddress} helper function checks if the supplied  is part of the current network by using ruby's \texttt{ipaddr} build in module. 

The file \texttt{lib/vm.rb} contains the code for managing the virtual machine instances, the class is called \texttt{Deeploy::Configurable::VM}. The class inherits from \texttt{Configurable}. Calling \texttt{VM.new} contains validators that raise errors upon initialising the class inappropriately. 
The method is responsible for specifying the distribution, available resources, configuration destination, name of the instance, packages, open ports and prepares it for creation. A design decision was made to allow a machine to be configured with the \texttt{new} operator and then created by issuing \texttt{machine\_instance.build()}.

The method \texttt{VM.alive()} verifies if the machine is alive by trying to listen on \gls{ssh} port, if it fails within a time-out, the machine is updated to state of not alive.	

\texttt{VM.get()} is used when performing power up, power down and destroy virtual machines. It queries the database and discovers everything about the instance, from packages, to \gls{ram} and location of configuration files.

The virtual machine is built only when an instance is initialised properly by calling \texttt{VM.new} with the correct arguments and then invoke \texttt{VM.build()}. The design decision to have a separate functionality of bootstrapping instances was mainly for cleaner testing by compartmentalising the components into smaller elements. The \texttt{build} method also accepts a boolean argument. This argument tells the current build if it is in "dry-run" mode. The mode, the machine will create directories and configuration without actually bringing the instance up and running. This is again, done for testing reasons - they verify that different \glspl{distribution} have appropriate configurations.

During the invocation of the \texttt{build} method, a \gls{bash} shell is forked and it executes the \gls{vagrant} command that brings it on-line and installs all dependencies. The shell command also tells the script to write all output to a text file in the machine directory, the file is called \texttt{vagrant.log}. Because the process takes a long time to finish, an axillary function \texttt{_wait_on_build} is used. This function constantly goes through the contents of the log file and checks what is the current building state of the instance. Typical states are building the base, installing puppet, setting up dependencies, installing packages and the state everything has finished.


TALK ABOUT CHDIR ISSUE THREAD SAFE AND ABOUT HEALTH MONITORING DEAMON INSTEAD OF CHECKING EVERYTIME DUE TO THE SLOWDOWN

\bibliography{work}
\renewcommand{\bibname}{}

\end{document}

\begin{filecontents*}{work.bib}
    @online{ruby-official-guide,
        author = {RubyGem Team},
        title = {Make your own gem},
        url = {http://guides.rubygems.org/make-your-own-gem/},
        urldate = {2011},
        note = {Accessed: April 13, 2017}
    }

    @online{netmask-definition,
        author = {Olivier Bonaventure},
        title = {Computer Networking : Principles, Protocols and Practice},
        url = {https://www.saylor.org/site/wp-content/uploads/2012/02/Computer-Networking-Principles-Bonaventure-1-30-31-OTC1.pdf},
        urldate = {2011},
        note = {Accessed: April 13, 2017}
    }
    
\end{filecontents*}
